\chapter{Conclusioni}
Il \textit{\textbf{Quantum Annealing}} rappresenta una delle applicazioni più promettenti della computazione quantistica nel campo dell’ottimizzazione combinatoria. Basandosi su fenomeni quali l'effetto tunnel e il Teorema Adiabatico, questo metodo consente di esplorare lo spazio delle soluzioni con un'efficienza unica, superando le limitazioni degli algoritmi classici e riducendo il rischio di rimanere nei minimi locali del paesaggio energetico.

Uno degli aspetti più significativi del \textit{Quantum Annealing} è la sua capacità di sfruttare l’effetto tunnel per superare le barriere energetiche che separano i minimi locali dal minimo globale. Questo approccio, unito alla flessibilità nella rappresentazione dei problemi e alla possibilità di implementare vincoli complessi, lo rende una tecnologia versatile per applicazioni in ambiti come la \textbf{logistica}, l’\textbf{intelligenza artificiale}, la \textbf{bioinformatica} e l’\textbf{analisi finanziaria}.

Sebbene l'implementazione pratica presenti ancora sfide significative, come la riduzione del gap minimo e il miglioramento della scalabilità nei dispositivi quantistici, i progressi tecnologici nel design di architetture hardware, come il grafo Pegasus, e nelle tecniche di embedding aprono nuove prospettive per un futuro in cui il Quantum Annealing potrà essere utilizzato per risolvere problemi complessi che oggi risultano inaccessibili ai calcolatori classici. 

Il Quantum Annealing non rappresenta solo un nuovo paradigma computazionale, ma anche un esempio di come i principi fondamentali della fisica quantistica possano essere sfruttati per risolvere problemi reali. Questo dimostra che l’innovazione tecnologica non si limita a spingere i confini della conoscenza teorica, ma contribuisce attivamente a trasformare le nostre capacità di affrontare e risolvere problemi complessi. Con il continuo avanzamento della ricerca, il Quantum Annealing potrebbe diventare un pilastro fondamentale nella prossima \textbf{rivoluzione computazionale}. 