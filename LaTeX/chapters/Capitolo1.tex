\chapter{Introduzione}
La \textbf{computazione quantistica} rappresenta una delle più affascinanti frontiere della scienza moderna, un crocevia dove fisica teorica, matematica e informatica si incontrano per sfidare i limiti del calcolo tradizionale. Sin dall'invenzione dei primi computer, avvenuta a metà del XX secolo, l'uomo ha cercato di costruire macchine sempre più veloci e potenti, in grado di risolvere problemi complessi in tempi ridotti. Tali macchine, dai primi calcolatori meccanici fino ai moderni supercomputer, hanno però sempre operato entro i vincoli imposti dai principi della fisica classica. È nel regno quantistico che si dischiudono nuove possibilità: fenomeni come l'entanglement, la sovrapposizione e il tunneling promettono di ridefinire radicalmente il concetto stesso di elaborazione dell'informazione.

Il percorso che ha portato alla nascita della computazione quantistica affonda le radici nel lavoro pionieristico di figure come \textit{Richard Feynman} e \textit{David Deutsch}. Negli anni '80, Feynman intuì che un computer basato su principi quantistici avrebbe potuto simulare sistemi fisici quantistici in modo più efficiente rispetto ai computer classici. Successivamente, Deutsch formalizzò questa intuizione introducendo il concetto di "computer quantistico universale", una macchina teorica in grado di eseguire qualsiasi algoritmo classico e, potenzialmente, di risolvere alcuni problemi in modo esponenzialmente più veloce.

Alla base di questa rivoluzione vi è il \textbf{quantum bit} (o in forma abbreviata \textbf{qubit}) unità fondamentale dell'informazione quantistica, il cui comportamento trascende la dicotomia binaria del bit. Il bit, nel paradigma classico, può esistere esclusivamente nello stato 1 o nello stato 0. Il qubit, invece, può esistere simultaneamente in più stati grazie al principio di sovrapposizione, aprendo la strada a forme di parallelismo computazionale prima impensabili. L'entanglement, una correlazione profonda e non locale tra qubit, permette di intrecciare informazioni in modi che sfidano l'intuizione umana e che trovano applicazioni tanto nella crittografia quanto nell'ottimizzazione. 

\pagebreak

Questa disciplina, un tempo confinata al regno della teoria, ha visto negli ultimi decenni un'accelerazione straordinaria. Dalla dimostrazione degli algoritmi di \textit{Shor} e \textit{Grover} negli anni '90, che hanno messo in evidenza il potenziale dei computer quantistici nel fattorizzare numeri primi e cercare elementi in un database non ordinato, fino ai progressi odierni nella realizzazione di computer quantistici scalabili, il campo della computazione quantistica non solo ridefinisce le capacità computazionali, ma sfida il confine tra il mondo classico e quello quantistico. 

I computer quantistici possono essere programmati utilizzando linguaggi di alto livello come \textbf{Python}, grazie a framework avanzati che traducono il codice in istruzioni comprensibili dall’hardware quantistico. Strumenti come \textbf{Qiskit}, \textbf{Cirq} e \textbf{Amazon Braket} consentono ai programmatori di descrivere algoritmi quantistici, progettare circuiti e interagire con dispositivi quantistici o simulatori. Sebbene Python non venga eseguito direttamente sull’hardware quantistico, funge da interfaccia per inviare comandi e analizzare i risultati. L’architettura sottostante traduce il codice in linguaggi a basso livello, come \textbf{QASM}, eseguibili dai processori quantistici. Questa metodologia permette di sfruttare la potenza della computazione quantistica mantenendo la semplicità e la leggibilità tipiche dei linguaggi di programmazione classici.

Due sono i principali paradigmi che si sono affermati per sfruttare le peculiarità della meccanica quantistica: il \textbf{Gate-Based Quantum Computing} e l'\textbf{Adiabatic Quantum Computing}. Sebbene entrambi condividano la capacità di elaborare l'informazione sfruttando fenomeni come la sovrapposizione e l'entanglement, i loro approcci concettuali e implementativi sono profondamente differenti, riflettendo due visioni distinte di come un sistema quantistico possa essere manipolato per risolvere problemi computazionali. 

L'approccio più studiato, e quello che attualmente appare come il futuro predominante della computazione quantistica, è il \textit{Gate-Based Quantum Computing}. Questo modello gode di una maggiore flessibilità teorica e applicativa, oltre a essere sostenuto da un ampio consenso nella comunità scientifica e da investimenti significativi da parte delle principali aziende tecnologiche. Tuttavia questo approccio non è esente da problematiche. Il principale ostacolo è rappresentato dalla \textbf{decoerenza quantistica}, il fenomeno per cui l'interazione di un sistema quantistico con l'ambiente esterno porta alla perdita di informazioni quantistiche. Poiché i qubit sono estremamente sensibili a perturbazioni esterne (come fluttuazioni termiche o elettromagnetiche), il loro stato quantistico tende a collassare rapidamente in uno stato classico, interrompendo i calcoli e riducendo la fedeltà dei risultati. Il fenomeno diventa critico nelle operazioni con porte quantistiche, che necessitano di coerenza temporale significativamente maggiore rispetto ai tempi di calcolo. 

\pagebreak

Il principale argomento di cui tratteremo in questa tesi sarà il \textbf{\textit{Quantum Annealing}}, una forma specializzata dell'\textit{Adiabatic Quantum Computing} progettata specificamente per risolvere problemi di ottimizzazione. A differenza dell'\textit{Adiabatic Quantum Computing}, che si propone come paradigma computazionale universale \cite{Label0}, il \textit{Quantum Annealing} è ottimizzato per individuare il minimo globale di una funzione di costo definita in uno spazio di soluzioni discreto.   

Nasce come estensione del \textbf{Simulated Annealing}, una tecnica classica di ottimizzazione introdotta nel 1983 da \textit{Kirkpatrick}, \textit{Gelatt} e \textit{Vecchi}, come un metodo per risolvere problemi di ottimizzazione combinatoria utilizzando analogie con i processi fisici di ricottura dei metalli. 

Il \textit{Quantum Annealing} fu formalizzato come concetto nel 1998 grazie al lavoro di \textit{T. Kadowaki} e \textit{H. Nishimori}, che dimostrarono la possibilità di utilizzare il \textbf{Teorema Adiabatico} per trovare il minimo energetico di un sistema \cite{Label0.5}. 

Negli anni 2000, si passò dalla teoria alla pratica con lo sviluppo di prototipi hardware per implementare il modello. In questo periodo, aziende come \textit{D-Wave Systems}, fondata nel 1999, iniziarono a costruire i primi dispositivi quantistici basati sul \textit{Quantum Annealing}.

Nel 2007, D-Wave presentò pubblicamente il primo computer quantistico commerciale, il \textit{D-Wave One}, in grado di risolvere problemi di ottimizzazione combinatoria utilizzando il \textit{Quantum Annealing}.

Il \textit{Quantum Annealing} si configura, quindi, come un approccio molto promettente, vista l'efficacia nella risoluzione dei problemi di ottimizzazione. Tuttavia, essendo una tecnica specializzata, esso resta meno versatile rispetto al \textit{Gate-Based Quantum Computing}, almeno per il momento. 

Le sfide tecnologiche ed epistemologiche che accompagnano la realizzazione dei computer quantistici rendono questo campo un simbolo del nostro tempo: un'era in cui la scienza e la tecnologia non solo spiegano il mondo, ma creano strumenti per plasmarlo. La computazione quantistica invita a ripensare profondamente le nostre idee su ciò che è possibile calcolare, aprendo nuovi orizzonti per affrontare problemi che la computazione classica non può risolvere in tempi ragionevoli. 


