\chapter{Quantum Bit}
\section{Dal Bit al Qubit}
Nella computazione classica, il \textbf{bit} è l'unità fondamentale dell'informazione e può assumere esclusivamente uno dei due stati definiti: 0 o 1. In ambito quantistico, l'analogo del bit è il \textbf{quantum bit}, o \textbf{qubit}, un'entità che sfrutta le proprietà della meccanica quantistica per rappresentare e processare informazioni. 

A differenza del bit classico, il qubit non è limitato a due stati distinti, ma può esistere in una sovrapposizione di entrambi gli stati base. Formalmente, lo stato di un qubit \cite{Label5} è descritto come: 
\begin{equation*}
	\lvert \psi \rangle = \alpha \lvert 0 \rangle + \beta \lvert 1 \rangle
\end{equation*}

Dove:
\begin{itemize}
	\item $\lvert 0 \rangle$ e $\lvert 1 \rangle$ rappresentano gli stati base del qubit rappresentati nella notazione di Dirac tramite i ket;
	\item $\alpha$ e $\beta$ sono numeri complessi che rappresentano l'ampiezza di probabilità degli stati $\lvert 0 \rangle$ e $\lvert 1 \rangle$, rispettivamente. 
\end{itemize}

Le ampiezze devono rispettare la condizione $|\alpha|^2 + |\beta|^2 = 1$, che garantisce che la somma delle probabilità sia sempre pari a 1. 

La rappresentazione dello stato del qubit si inserisce nel contesto del \textbf{primo postulato della meccanica quantistica}, secondo cui un sistema quantistico è descritto da un vettore di stato appartenente a uno spazio di Hilbert complesso. Questo vettore contiene tutte le informazioni relative al qubit, inclusa la probabilità di osservare specifici stati base durante una misurazione. 

La realizzazione fisica di un qubit richiede l'impiego di un sistema quantistico che possa essere manipolato per rappresentare, controllare e misurare stati quantistici. Una delle implementazioni fisiche più comuni si basa sullo spin elettronico, ovvero il momento angolare intrinseco associato a un singolo elettrone. Gli elettroni sono intrappolati in strutture di semiconduttori, che sfruttano la tecnologia già esistente nel mondo dei chip classici.  In questo contesto, il qubit è codificato nei due possibili orientamenti dello spin dell'elettrone. 

Gli stati $\lvert 0 \rangle$ e $\lvert 1 \rangle$ sono rappresentati, rispettivamente, dagli stati di spin "up" ($\uparrow$) e spin "down" ($\downarrow$). Lo spin può essere manipolato tramite campi magnetici o impulsi elettrici. Sebbene siano possibili diverse implementazioni fisiche dei qubit, nel contesto della presente analisi sul \textit{Quantum Annealing} si adotterà quella basata sullo spin elettronico. 

\section{Proprietà del Qubit}
Il qubit possiede proprietà intrinseche che derivano direttamente dai principi fondamentali della meccanica quantistica e che meritano un'analisi più approfondita.

\subsection{Sovrapposizione}
La prima proprietà fondamentale del qubit, l'abbiamo già introdotta ed è la \textbf{sovrapposizione} \cite{Label6}. Questa proprietà consente a un singolo qubit di rappresentare un insieme più ampio di informazioni rispetto a un bit classico, grazie alla possibilità di trovarsi in una sovrapposizione degli stati base 0 e 1. Tuttavia ciò non implica che sia possibile accedere simultaneamente a entrambi gli stati base durante un'operazione di misurazione. 

In base al \textbf{terzo postulato della meccanica quantistica}, la misurazione di un qubit proietta lo stato quantistico in uno dei suoi stati base, distruggendo la sovrapposizione preesistente. In particolare, il risultato della misurazione sarà $0$ con probabilità $|\alpha|^2$, o $1$ con probabilità $|\beta|^2$. 

Un esempio classico di stato di sovrapposizione di un qubit, fondamentale nel calcolo quantistico, è il seguente:
\begin{equation*}
	\lvert \psi \rangle = \frac{\lvert 0 \rangle +\lvert 1 \rangle}{\sqrt{2}}
\end{equation*}
Questo stato è noto come stato di sovrapposizione equiprobabile ed è una combinazione lineare in cui il qubit ha una probabilità del $50\%$ di collassare nello stato $\lvert 0 \rangle$ e una probabilità del $50\%$ di collassare nello stato $\lvert 1 \rangle$  quando viene misurato nella base computazionale. Il fattore $\frac{1}{\sqrt{2}}$ è necessario per soddisfare la condizione di normalizzazione: $|\alpha|^2 + |\beta|^2 = 1$. 

\pagebreak 

\subsection{Entanglement}
L'\textbf{entanglement} è una delle proprietà più straordinarie della meccanica quantistica, che si manifesta quando due o più qubit vengono correlati in un modo che il loro stato complessivo non può essere descritto indipendentemente. Quando due o più qubit sono in uno stato di entanglement, lo stato di ciascun qubit dipende istantaneamente dallo stato dell'altro, indipendentemente dalla distanza che li separa, grazie alla natura non-locale della meccanica quantistica. Questo fenomeno, descritto da \textit{Einstein} come "spooky action at a distance", non ha equivalenti nel mondo classico. 

Per comprendere meglio questo fenomeno, consideriamo il caso di due qubit descritti da uno spazio di Hilbert, che include tutti i possibili stati combinati. Gli stati base di due qubit possono essere scritti come: 
\begin{align*}
	\lvert 00 \rangle \text{  } \lvert 01 \rangle \text{  } \lvert 10 \rangle \text{  } \lvert 11 \rangle 
\end{align*}

Qualsiasi stato sovrapposto può essere descritto come:
\begin{equation*}
\lvert \psi \rangle = \alpha \lvert 00 \rangle + \beta \lvert 01 \rangle + \gamma \lvert 10 \rangle + \delta \lvert 11 \rangle
\end{equation*}

Tuttavia, nel caso dell'entanglement, lo stato complessivo del sistema non può essere scritto come un semplice prodotto di stati di ciascun qubit. Lo abbiamo visto con il \textbf{quarto postulato della meccanica quantistica}. 

Gli \textbf{stati di Bell} rappresentano un esempio classico di stati entangled e sono definiti come stati di sovrapposizione massimale tra due qubit. Essi costituiscono il paradigma fondamentale per lo studio dell'entanglement quantistico. Gli stati di Bell, che sono i seguenti, rappresentano i più semplici esempi di stati entangled: 
\\\\
\begin{equation*}
	\lvert \Phi^+ \rangle = \frac{1}{\sqrt{2}} \left( \lvert 00 \rangle + \lvert 11 \rangle \right)
\end{equation*}
\begin{equation*}
	\lvert \Phi^- \rangle = \frac{1}{\sqrt{2}} \left( \lvert 00 \rangle - \lvert 11 \rangle \right)
\end{equation*}
\begin{equation*}
	\lvert \Psi^+ \rangle = \frac{1}{\sqrt{2}} \left( \lvert 01 \rangle + \lvert 10 \rangle \right)
\end{equation*}
\begin{equation*}
	\lvert \Psi^- \rangle = \frac{1}{\sqrt{2}} \left( \lvert 01 \rangle - \lvert 10 \rangle \right)
\end{equation*}
\\

Quando si misura un sistema entangled, il processo di misurazione di un qubit influenza direttamente lo stato del secondo qubit. 

\pagebreak

In uno stato di Bell come \( \lvert \Phi^+ \rangle \) o \( \lvert \Phi^- \rangle \), i due qubit sono "allineati", ossia misurare uno stato come \( \lvert 0 \rangle \) implica che l'altro sarà misurato come \( \lvert 0 \rangle \), e se uno è \( \lvert 1 \rangle \), anche l'altro lo sarà. 
In uno stato come \( \lvert \Psi^+ \rangle \) o  \( \lvert \Psi^- \rangle \), i due qubit sono in stati opposti: se misuriamo uno come \( \lvert 0 \rangle \), l'altro sarà \( \lvert 1 \rangle \) e viceversa. 

In pratica gli stati di Bell descrivono come due qubit possono essere entangled. 

\subsection{Interferenza}
I qubit possono anche manifestare il fenomeno dell'interferenza, che si verifica quando le ampiezze di probabilità degli stati si combinano in modi che possono aumentare o ridurre la probabilità di determinati risultati. Essendo l'equazione di Schrödinger un'equazione d'onda, l'interferenza può essere sfruttata per: 
\begin{itemize}
	\item Ridurre la probabilità che lo stato finale non sia indesiderato (\textbf{interferenza quantistica distruttiva});
	\item Aumentare la probabilità di trovare i qubit su uno stato finale che codifica la soluzione al problema dato (\textbf{interferenza quantistica costruttiva}).
\end{itemize}
Le tecniche di interferenza sono utilizzate nei circuiti quantistici per manipolare gli stati dei qubit, rendendo alcune configurazioni più probabili di altre. Questo permette di amplificare le probabilità degli stati desiderati e ridurre quelle degli stati indesiderati, migliorando l'efficienza degli algoritmi quantistici. 

\subsection{Effetto Tunnel}
Un'altra proprietà fondamentale del qubit è l'\textbf{effetto tunnel}, che riveste un ruolo cruciale nel contesto del \textit{Quantum Annealing}. L'effetto tunnel consente a un sistema quantistico, come quello associato a un qubit, di attraversare una barriera energetica anche quando, secondo la meccanica classica, la sua energia non sarebbe sufficiente a superarla. In un problema di ottimizzazione, ciò implica che, se una soluzione prossima al minimo globale è separata da una barriera energetica più alta rispetto a un minimo locale, il sistema quantistico può passare attraverso tale barriera, consentendo di raggiungere il minimo globale. 

\pagebreak

\section{Sfera di Bloch}
\begin{figure}[H]
	\centering
	\includegraphics[width= 0.5\textwidth]{images/Bloch_Sphere.png}  
	\label{fig:Sfera di Bloch}
\end{figure}

Il comportamento di un qubit può essere rappresentato graficamente sulla \textbf{Sfera di Bloch}, una rappresentazione geometrica che ne illustra l'orientamento in uno spazio tridimensionale \cite{Label7}. La sfera ha raggio unitario, con gli stati base $\lvert 0 \rangle $ e $\lvert 1 \rangle $ situati rispettivamente ai poli nord e sud della sfera, mentre gli stati sovrapposti sono rappresentati da punti sulla superficie della sfera, dove ciascun punto corrisponde a una possibile combinazione degli stati base.  

Ogni stato del qubit può essere descritto nella forma parametrica: 
\begin{equation*}
|\psi\rangle = \cos\left(\frac{\theta}{2}\right) |0\rangle + e^{i\varphi} \sin\left(\frac{\theta}{2}\right) |1\rangle
\end{equation*}
Dove:
\begin{itemize}
	\item $\theta$ è l'angolo di latitudine (o angolo zenitale) che misura la distanza angolare dallo stato $\lvert 0 \rangle$ (polo nord);
	\item $\varphi$ è l'angolo di longitudine (o angolo azimutale) che determina la fase relativa tra gli stati $\lvert 0 \rangle$ e $\lvert 1 \rangle$.
\end{itemize}
Pertanto, un generico vettore di stato $\lvert \psi \rangle$ è rappresentato sulla Sfera di Bloch e può essere caratterizzato da un vettore unitario, individuato dai due angoli $\theta$ e $\varphi$. Questa rappresentazione geometrica aiuta a visualizzare la manipolazione e l'evoluzione di un qubit in un sistema quantistico, specialmente quando si applicano operazioni (ad esempio con porte quantistiche) che modificano l'orientamento del qubit sulla sfera di Bloch.

La sfera di Bloch visualizza anche il processo di misurazione. In base al \textbf{terzo postulato}, una misurazione lungo un asse specifico provoca il collasso dello stato del qubit in uno degli stati base associati a quell'asse.

\section{Vantaggi computazionali del Qubit}
La capacità dei qubit di esistere in più stati contemporaneamente, combinata con fenomeni come l'entanglement e l'interferenza, conferisce ai sistemi quantistici un'enorme potenza computazionale. 

In un sistema classico, per rappresentare $n$ bit sono necessarie $2^{n} $ possibili combinazioni di stati, ma solo uno di questi stati può essere rappresentato e processato in un dato momento. Questo limita la capacità di esplorare soluzioni simultaneamente, costringendo il sistema a una ricerca sequenziale. 

In un sistema quantistico, invece, $n$ qubit possono rappresentare simultaneamente una sovrapposizione di $2^{n} $ stati distinti. Grazie a ciò, un singolo qubit può trovarsi in una combinazione di entrambi gli stati base ($\lvert 0 \rangle$ e $\lvert 1 \rangle$), e quindi un sistema di $n$ qubit può rappresentare simultaneamente tutte le possibili configurazioni di n bit. Ciò implica che un computer quantistico con un numero relativamente ridotto di qubit ha il potenziale di esplorare uno spazio di soluzioni esponenzialmente più vasto rispetto a un computer classico.

Inoltre, grazie all'entanglement quantistico, i qubit non solo esistono in sovrapposizione, ma possono anche essere correlati in modi che non hanno analoghi nei sistemi classici. Questa interazione permette a un computer quantistico di eseguire operazioni su più stati contemporaneamente, rendendo possibili calcoli molto più complessi in tempi significativamente più brevi.

Pertanto, un algoritmo quantistico può risolvere problemi complessi, come la fattorizzazione di numeri grandi o la ricerca in spazi di soluzione molto ampi, molto più velocemente di quanto potrebbe fare un algoritmo classico. 

Un esempio concreto del potenziale dei computer quantistici è la loro capacità di generare numeri casuali. Nei computer classici, la generazione di numeri casuali è generalmente realizzata tramite algoritmi deterministici noti come generatori di numeri pseudo-casuali. Tali algoritmi, a partire da un valore iniziale chiamato "\textit{seme}" (\textbf{seed}), producono una sequenza di numeri che simula la casualità, ma che in realtà è completamente deterministica. Ciò significa che, dato lo stesso seme, verrà sempre generata la stessa sequenza di numeri. 

\pagebreak

Nei computer quantistici, invece, la generazione di numeri casuali può essere intrinseca al processo stesso, grazie alla natura probabilistica della meccanica quantistica. Questo processo è fondamentalmente non deterministico, poiché il risultato della misurazione è casuale e non può essere previsto con certezza. Sebbene la generazione di numeri casuali possa sembrare un argomento banale, essa riveste un'importanza cruciale, poiché costituisce la base per numerosi algoritmi di crittografia. 

Nonostante la potenza computazionale offerta dai qubit, i bit classici rimangono fondamentali nei computer quantistici. Un problema significativo nel contesto del calcolo quantistico è rappresentato dalla \textbf{decoerenza quantistica}, che si riferisce alla perdita di coerenza degli stati quantistici a causa di interazioni con l'ambiente esterno, come fluttuazioni termiche o l'influenza di campi elettromagnetici. Ad esempio, una piccola fluttuazione termica può generare l'emissione di un fotone a bassa energia, che interagendo con il qubit ne altera lo stato e ne compromette l'affidabilità del risultato del calcolo.

Sebbene il processo di elaborazione avvenga tramite qubit, l'output finale deve essere convertito in bit classici. Questa conversione è necessaria anche perché i dispositivi di output, come memorie e schermi, operano nel dominio classico e non sono in grado di interpretare direttamente gli stati quantistici. Pertanto, i bit classici svolgono un ruolo cruciale nel rendere i risultati delle operazioni quantistiche comprensibili e utilizzabili nell'ambito delle tecnologie computazionali tradizionali. 


