\chapter{Quantum Annealing per il Problema del Commesso Viaggiatore}
\section{Problema del Commesso Viaggiatore}
Il \textbf{problema del Commesso Viaggiatore} (\textit{Traveling Salesman Problem}, \textbf{TSP}) è un classico problema di ottimizzazione combinatoria \cite{Label13}. Esso richiede di determinare il percorso più breve che permette a un commesso viaggiatore di visitare un insieme di città, passando per ciascuna esattamente una volta, per poi tornare al punto di partenza. 

Il TSP è noto per la sua rilevanza sia teorica che pratica, ed è classificato come problema \textit{NP-hard}, il che implica che non esistono algoritmi noti in grado di risolverlo in tempo polinomiale per ogni caso possibile (a meno che $P=NP$, ma ricordo che determinarlo è uno dei \textit{problemi del millennio}). 


Può essere descritto matematicamente come segue:
\begin{itemize}
	\item Siano $G = (N, A)$ un grafo completo, $N = \{1, 2, \dots, n\}$ l’insieme delle città e $A$ l’insieme degli archi; 
	\item A ciascun arco $(i; j) \in A$ è associato un costo $c_{ij}$, che rappresenta la distanza o il peso del collegamento tra le città $i$ e $j$.
\end{itemize}

Un \textbf{ciclo hamiltoniano} è un percorso in un grafo che visita ciascun nodo esattamente una volta e ritorna al nodo di partenza, formando così un ciclo chiuso. L'obiettivo del TSP è determinare un ciclo hamiltoniano $x= (x_1, x_2, \dots, x_n)$, dove ogni città è visitata esattamente una volta, minimizzando la somma totale dei costi associati al percorso. 

Il TSP è \textbf{simmetrico} se \( c_{ij} = c_{ji} \ \forall (i; j) \), altrimenti si dice \textbf{asimmetrico}. 


\pagebreak

\( x_{ij} \) è la generica variabile binaria del problema. In pratica \( x_{ij} = 1 \) se l'arco \((i; j)\) appartiene al circuito, altrimenti \( x_{ij} = 0 \). Una possibile formulazione matematica del problema è:
\begin{equation*}
	\begin{aligned}
		& min(\sum_{i=1}^{N} \sum_{j=1}^{N} c_{ij} x_{ij}), \\
		& \sum_{j=1, j \neq i}^{N} x_{ij} = 1 \quad \forall i \in N, \\
		& \sum_{i=1, i \neq j}^{N} x_{ij} = 1 \quad \forall j \in N, \\
		& \sum_{(i,j)\in A, i \in S} \sum_{j \notin S} x_{ij} \geq 1 \quad \forall S \subset N, S \neq 0, S \neq N, \\
		& x_{ij} \in \{0, 1\} \quad \forall i, j \in N.
	\end{aligned}
\end{equation*}
I primi due tipi di vincoli mi assicurano che ogni città sia visitata una sola volta e sono detti \textbf{vincoli di assegnazione}. Il primo nello specifico mi dice che ogni città deve essere lasciata una sola volta, il secondo che ogni città deve essere raggiunta una sola volta. 

Il terzo tipo di vincoli evita cicli interni che non coprono tutte le città, garantendo un unico ciclo hamiltoniano e l'assenza di sottocicli. Sono detti \textbf{vincoli di connessione}. In particolare essi definiscono che, comunque si scelga un sottoinsieme proprio di nodi $S$, deve esistere almeno un arco che colleghi un nodo di $S$ con un nodo non appartenente a $S$.

\section{TSP nella forma QUBO}
Per risolvere il problema tramite il \textit{Quantum Annealing}, è necessario prima rappresentarlo in forma \textbf{QUBO}. Tuttavia, il TSP, nella sua formulazione classica, è difficile da esprimere in questa forma.

Questo limite deriva dalla difficoltà di rappresentare esplicitamente i \textbf{vincoli di connessione}. La modellazione di tali vincoli spesso richiede espressioni matematiche più complesse di quelle quadratiche e comporta un aumento significativo del numero di variabili, rendendo il problema inefficiente da risolvere.

Nei metodi classici, come l'ottimizzazione lineare, questi vincoli vengono gestiti iterativamente durante il processo di ottimizzazione. Nel caso del QUBO, invece, tutti i vincoli devono essere incorporati direttamente nella funzione obiettivo sotto forma di \textbf{penalità} e/o \textbf{ricompense}. 

\pagebreak

Rappresentare in modo efficace vincoli complessi, come quelli relativi agli archi nel TSP, risulta quindi impraticabile senza introdurre semplificazioni o approssimazioni. Per superare queste difficoltà, è possibile adottare una formulazione alternativa del problema, basata \textbf{sull'ordine di visita delle città}. 

In questa nuova rappresentazione, il problema viene modellato utilizzando variabili binarie $x_{ip}$ che assumono il valore 1 se una città $i$ occupa una posizione $p$ lungo il percorso e 0 altrimenti. La funzione obiettivo e i vincoli del problema si possono così esprimere nella forma seguente:
\begin{equation*}
	\begin{aligned}
		& min(\sum_{i=1}^{N} \sum_{j=1}^{N} \sum_{p=1}^{N} c_{ij} \cdot x_{ip} \cdot x_{j(p+1)}), \\
		& \sum_{p=1}^{N} x_{ip} = 1 \quad \forall i \in N, \\
		& \sum_{i=1}^{N} x_{ip} = 1 \quad \forall p = \{1, \dots, n\}, \\
		& x_{ip} \in \{0, 1\} \quad \forall i \in N, \forall p = \{1, \dots, n\}.
	\end{aligned}
\end{equation*}
dove:
\begin{itemize}
	\item $c_{ij}$ rappresenta la distanza (il costo) tra le città $i$ e $j$; 
	\item Il primo tipo di vincoli assicura che ogni città $i$ sia visitata esattamente una volta e occupi una posizione unica nel percorso;
	\item Il secondo tipo di vincoli garantisce che ogni posizione $p$ nel percorso sia assegnata a una sola città.
\end{itemize}

Inoltre, per completare il ciclo, si impone $x_{j,n+1} = x_{j,1}$ per tornare al punto di partenza. 

Per passare dal problema del TSP alla formulazione QUBO, costruiamo una matrice Q che incorpora la funzione obiettivo e i vincoli sotto forma di penalità e ricompense. 

La \textbf{ricompensa} $R$ è un numero positivo, il cui valore deve essere maggiore della distanza più grande, mentre la \textbf{penalità} $P$ è un numero positivo, che deve essere maggiore o uguale alla ricompensa. Questo requisito garantisce che le penalità associate alla violazione dei vincoli prevalgano sui contributi positivi della ricompensa nella funzione obiettivo.

Sebbene la ricompensa sia un valore positivo, essa viene successivamente sottratta nella formulazione matematica, contribuendo così a bilanciare l'influenza del termine di ottimizzazione e dei vincoli. 

\pagebreak

La matrice $Q$ del QUBO viene aggiornata con i costi effettivi se due città $i$ e $j$ occupano posizioni adiacenti:
\begin{equation*}
	Q_{ip,jq}= c_{ij} \, \quad \text{se } i \neq j \text{ e } q=p+1
\end{equation*}

Se una città $i$ appare in due posizioni diverse $p$ e $q$, questa combinazione è vietata:  
\begin{equation*}
	Q_{ip,jq}= P \, \quad \text{se } i=j \text{ e } p \neq q
\end{equation*}

Se due città $i$ e $j$ occupano la stessa posizione $p$, questa combinazione è vietata:
\begin{equation*}
	Q_{ip,jq}= P \, \quad \text{se } i \neq j \text{ e } p=q
\end{equation*}

Infine per evitare che la soluzione imponga la stessa città su più posizioni, togliamo la ricompensa sulla diagonale: 
\begin{equation*}
	Q_{ip,jq} = -R \, \quad \text{se } i = j \text{ e } p = q
\end{equation*}

Se non rientra in nessuno dei precedenti casi:
\begin{equation*}
	Q_{ip,jq}= 0
\end{equation*}\\

Riassumendo: 
\begin{equation*}
	Q_{ip,jq} = 
	\begin{cases} 
		Q_{ip,jq}= c_{ij} & \quad \text{se } i \neq j \text{ e } q=p+1 \text{;}\\
		Q_{ip,jq}= P & \quad \text{se } i=j \text{ e } p \neq q \text{;}\\
		Q_{ip,jq}= P & \quad \text{se } i \neq j \text{ e } p=q \text{;}\\
		Q_{ip,jq} = -R & \quad \text{se } i = j \text{ e } p = q \text{;}\\
		Q_{ip,jq}= 0 & \text{ altrimenti}
	\end{cases}
\end{equation*}

Non entreremo nel dettaglio della scelta dei termini $P$ ed $R$, in quanto questa parte del processo dipende da considerazioni specifiche che esulano dallo scopo di questo lavoro. Tuttavia, è importante sottolineare che, generalmente, il calcolo di questi termini si basa sulla dimensione del problema e sui pesi delle connessioni del grafo.

\pagebreak

\section{Progettazione Algoritmo}
La \textbf{progettazione di un algoritmo di Quantum Annealing} segue una serie di passi strutturati che permettono di tradurre un problema di ottimizzazione in un modello risolvibile tramite il quantum annealer. 

La prima fase consiste nella \textbf{definizione del problema}. Ogni problema deve essere espresso come una funzione obiettivo che rappresenti un costo da minimizzare o una qualità da massimizzare. Il \textit{Quantum Annealing} è progettato per risolvere problemi di minimo, quindi se si ha un problema di massimo si applica questa formula alla funzione obiettivo:
\begin{equation*}
	max(f(x)) = - min(-f(x))
\end{equation*}
I vincoli del problema rimangono invariati durante questa conversione. Se sono presenti penalità associate ai vincoli, anch'esse restano additive nella funzione obiettivo trasformata. Le ricompense restano sottrattive.  

Successivamente, si passa alla \textbf{rappresentazione della soluzione}. Nel \textit{Quantum Annealing} è determinata dal numero di qubit necessari per modellare tutte le possibili configurazioni del problema. Ogni qubit rappresenta una decisione binaria o uno stato del sistema. Bisogna porre particolare attenzione a questo step, perché è necessario per rendere più facile la costruzione della matrice QUBO. 

La fase seguente riguarda la \textbf{definizione dei vincoli}, che deve garantire che le soluzioni candidate rispettino le regole del problema. Questi vincoli vengono tradotti in termini matematici attraverso penalizzazioni e/o ricompense inseriti nella funzione obiettivo. 

Una volta definiti i vincoli, è necessario \textbf{costruire la matrice QUBO}, che rappresenta matematicamente il problema sotto forma di una matrice quadrata simmetrica. Gli elementi diagonali della matrice descrivono i costi o i premi associati alle decisioni individuali, mentre gli elementi non diagonali rappresentano le interazioni tra diverse decisioni. 

Dopo aver costruito la matrice QUBO, il problema viene \textbf{convertito nel modello di Ising}, che è direttamente interpretabile dai quantum annealer. La conversione mantiene la struttura e il significato matematico del problema. 

Il passo successivo è l’\textbf{embedding del modello di Ising nel grafo Chimera} o in una topologia hardware compatibile. Questo processo richiede un numero maggiore di qubit fisici rispetto a quelli logici, ma strumenti software specializzati semplificano questa operazione. 

\pagebreak

Una volta completato l’embedding, il quantum annealer esegue la \textbf{determinazione del minimo di energia} attraverso il processo di \textit{Quantum Annealing}. È qui che il sistema viene inizializzato nello stato fondamentale di un Hamiltoniano iniziale e successivamente evoluto verso lo stato fondamentale dell’Hamiltoniano del problema. L’adiabaticità del processo garantisce che il sistema rimanga nello stato di energia minima, a meno di interruzioni causate da decoerenza o transizioni non adiabatiche.  

Infine, si giunge all’\textbf{interpretazione della soluzione}, in cui lo stato finale del sistema viene decodificato per estrarre la configurazione ottimale delle variabili originali. La soluzione restituita è nel modello di Ising e viene poi convertita in forma QUBO in modo che sia più facile da interpretare. 

Riassumendo, i passaggi per la progettazione di un algoritmo \cite{Label14} sono: 
\begin{enumerate}
	\item \textbf{Definizione del Problema};
	\item \textbf{Rappresentazione della Soluzione};
	\item \textbf{Definizione dei Vincoli};
	\item \textbf{Costruzione della matrice QUBO};
	\item \textbf{Conversione nel Modello di Ising};
	\item \textbf{Embedding con grafico Chimera};
	\item \textbf{Determinazione del minimo di energia};
	\item \textbf{Interpretazione della soluzione}.
\end{enumerate}

\pagebreak

\section{Esempio pratico di risoluzione di un TSP}
Per illustrare l'applicazione del \textit{Quantum Annealing}, consideriamo un esempio pratico basato su un problema del Commesso Viaggiatore simmetrico (rappresentato nella figura sotto). La rappresentazione grafica del problema da risolvere è la seguente:
\begin{figure}[H]
	\centering
	\includegraphics[width= 1.0\textwidth]{images/TSP.png} 
	\label{fig:TSP}
\end{figure}

La città A deve essere allo stesso tempo, la città di partenza e la città di arrivo. 

\subsection{Passo 1: Definizione del Problema}
Come discusso in precedenza, la formulazione classica del TSP, in cui l'obiettivo è determinare direttamente gli archi che compongono il percorso ottimale, risulta complessa da convertire in una forma compatibile con il modello QUBO. Pertanto proponiamo una formulazione alternativa del problema, focalizzandoci sull'\textbf{ordine delle città da visitare}. In questa rappresentazione, l'obiettivo è determinare la sequenza ottimale di città che minimizza il costo totale del percorso, rispettando i vincoli di unicità e ciclicità. 

\pagebreak

\subsection{Passo 2: Rappresentazione della Soluzione}
In un problema di TSP con $N$ città, è necessario garantire che ogni città sia assegnata a una posizione specifica nel percorso. Per rappresentare questa relazione nella formulazione QUBO, sono richiesti $N^2$ qubit logici: ciascuna città può occupare $N$ posizioni nel percorso e ogni combinazione è rappresentata da un qubit.

Nel nostro problema vi sono 4 città. Teoricamente, sarebbero necessari 16 qubit logici per rappresentare tutte le possibili assegnazioni. Tuttavia, è possibile ridurre il numero di qubit richiesti sfruttando le proprietà del problema e introducendo alcune ottimizzazioni.

Poiché in un TSP la città di partenza e di arrivo sono la stessa (in questo caso è A), è sufficiente escluderla dalla rappresentazione. Questo riduce il problema a determinare l'ordine delle altre $N-1$ città. Di conseguenza, con 4 città, il numero di qubit logici necessari si riduce a: 
\begin{equation*}
	(N-1)^2 = 3^2 = 9
\end{equation*}
Questa riduzione sfrutta la simmetria del problema e semplifica il modello, pur preservando la correttezza della rappresentazione e la capacità di determinare il percorso ottimale. 

\begin{figure}[H]
	\centering
	\includegraphics[width= 0.9\textwidth]{images/Rappresentazione_TSP.png} 
\end{figure}

\subsection{Passo 3: Definizione dei Vincoli}
Abbiamo già visto come si formulano i vincoli del TSP con la formulazione basata sull'ordine delle città. Essi possono essere riassunti come segue:
\begin{enumerate}
	\item \textbf{Ogni città deve essere visitata una sola volta};
	\item \textbf{Ogni posizione del percorso deve essere occupata da una sola città};
	\item \textbf{La distanza tra città consecutive deve influire sul costo totale del percorso}. 
\end{enumerate}

\pagebreak

La formalizzazione matematica di questi vincoli è la seguente: 
\begin{equation*}
	\begin{aligned}
		& min(\sum_{i \in \{B,C,D\}} \sum_{j \in \{B,C,D\}} \sum_{p=2}^{4} c_{ij} \cdot x_{ip} \cdot x_{ip+1}), \\
		& \sum_{p=2}^{4} x_{ip} = 1 \quad \forall i \in \{B,C,D\} \\
		& \sum_{i \in \{B,C,D\}} x_{ip} = 1 \quad \forall p = \{2,3,4\}\\
		& x_{ip} \in \{0, 1\} \quad \forall i \in \{B,C,D\}, \forall p = \{2,3,4\}.
	\end{aligned}
\end{equation*} 

\subsection{Passo 4: Costruzione della matrice QUBO}
Con queste semplificazioni, possiamo esprimere $Q$ nel seguente modo:
\begin{equation*}
	Q_{ip,jq} = 
	\begin{cases} 
		Q_{ip,jq}= c_{ij} & \quad \text{se } i \neq j \text{ e } q=p+1 \text{;}\\
		Q_{ip,jq}= P & \quad \text{se } i=j \text{ e } p \neq q \text{;}\\
		Q_{ip,jq}= P & \quad \text{se } i \neq j \text{ e } p=q \text{;}\\
		Q_{ip,jq} = c_{ij} - R & \quad \text{se } i = j \text{ e } p = q \text{ con } p = 2 \text{ o } p = N \text{;}\\
		Q_{ip,jq} = -R  & \quad \text{se } i = j \text{ e } p = q \text{ con } p \neq 2 \text{ e } p \neq N \text{;}\\
		Q_{ip,jq}= 0 & \text{ altrimenti}
	\end{cases}
\end{equation*}

Applicandola al nostro problema, troviamo la seguente matrice:
\begin{figure}[H]
	\centering
	\includegraphics[width= 0.6\textwidth]{images/QUBO_TSP.png}
	\label{fig:QUBO TSP} 
\end{figure}

\pagebreak

Il TSP è simmetrico e di conseguenza pure questa matrice lo è. Questo implica che i valori nella matrice siano speculari rispetto alla diagonale, ovvero i riquadri grigi sono identici per le coppie di città simmetriche.

\subsection{Passo 5: Conversione nel Modello di Ising}
Il modello QUBO viene tradotto in un modello di Ising per l’implementazione sul quantum annealer. Applicando l'algoritmo di conversione otteniamo la seguente matrice: 
\begin{figure}[H]
	\centering
	\includegraphics[width= 0.7\textwidth]{images/Ising_TSP.png}
	\label{fig:Ising TSP} 
\end{figure}

\subsection{Passo 6: Embedding con grafico Chimera}
Il modello di Ising può essere rappresentato come un grafo. In questa rappresentazione:
\begin{itemize}
	\item I vertici del grafo corrispondono ai qubit logici; 
	\item Un arco viene introdotto tra due vertici se il valore corrispondente nella matrice Q (nella forma di Ising) è diverso da zero.
\end{itemize}

\pagebreak

\begin{figure}[H]
	\centering
	\includegraphics[width= 1.1\textwidth]{images/Ising_Logical_Qubit.png}
	\label{fig:Da Ising al grafico che rappresenta i Qubit Logici} 
\end{figure}
Questo grafico ci rende più facile la comprensione del meccanismo di \textbf{embedding}. $N$ qubit logici richiedono $N^{2}$ qubit fisici nel caso peggiore. Il nostro modello di Ising deve essere mappato sulla topologia fisica del grafo Chimera. 

Utilizzando dei tool specifici è stato possibile ottenere un embedding che richiede 27 qubit fisici per rappresentare 9 qubit logici. Considerando che nel caso peggiore sarebbe stato necessario l'uso di 81 qubit fisici, il risultato ottenuto è soddisfacente. 

\begin{figure}[H]
	\centering
	\includegraphics[width= 1.1\textwidth]{images/Qubit_Logici_Fisici.png}
	\label{fig:Dal grafico che rappresenta i Qubit Logici al Grafo Chimera} 
\end{figure}

\pagebreak
\subsection{Passo 7: Determinazione del Minimo di Energia}
Il quantum annealer risolve il problema determinando lo stato fondamentale (minimo energetico) dell’Hamiltoniano totale. Il \textit{Quantum Annealing} sfrutta l’effetto tunnel per superare barriere energetiche, evitando di rimanere bloccato nei minimi locali e trovando il minimo globale.

\subsection{Passo 8: Interpretazione della soluzione}
Una volta identificata la configurazione dei qubit che corrisponde all'energia minima, il passo successivo consiste nel tradurre questa configurazione nella soluzione del problema originale. 

Nel caso del TSP, ciò implica determinare il percorso ottimale, ossia la sequenza di città che minimizza la distanza totale, risolvendo così il problema di ottimizzazione. 

Questa fase richiede l'interpretazione della configurazione dei qubit in termini di variabili logiche, seguendo la mappatura stabilita durante l'embedding, per ricostruire il percorso che rappresenta la soluzione migliore. \\\\

\begin{figure}[H]
	\centering
	\includegraphics[width= 1.0\textwidth]{images/Soluzione_TSP.png}
	\caption{Soluzione del Problema} 
	\label{Soluzione TSP} 
\end{figure}

\pagebreak

Ricostruita la soluzione per il modello di Ising, viene poi presentata in forma QUBO. 

La soluzione al nostro problema specifico è: $A \rightarrow C \rightarrow D \rightarrow B \rightarrow A$. \linebreak In formulazione QUBO si traduce in:

\begin{equation*}
	x = 
	\begin{aligned}
		\begin{bmatrix}
			0 \\
			0 \\
			1 \\
			1 \\
			0 \\
			0 \\
			0 \\
			1 \\
			0 \\
		\end{bmatrix}
	\end{aligned}
\end{equation*}








